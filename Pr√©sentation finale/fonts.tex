%%
%
%	Définition des polices utilisées pour le thème
%
%	--------------------
%	Distribué sous license GNU-Linux
%
%	This program is free software: you can redistribute it and/or modify
%   it under the terms of the GNU General Public License as published by
%   the Free Software Foundation, either version 3 of the License, or
%   (at your option) any later version.
%   This program is distributed in the hope that it will be useful,
%   but WITHOUT ANY WARRANTY; without even the implied warranty of
%   MERCHANTABILITY or FITNESS FOR A PARTICULAR PURPOSE.  See the
%   GNU General Public License for more details.
%   You should have received a copy of the GNU General Public License
%   along with this program.  If not, see <http://www.gnu.org/licenses/>.
%
%%

% Configuration et package pour charger des polices depuis le disque
\usepackage[no-math]{fontspec}
\defaultfontfeatures{Ligatures=TeX}
\frenchspacing

% Définition explicite des fonts. On les charge directement depuis le disque.
% Dans l'absolu, fontspec peut les chercher dans les fichiers de fonts de Windows,
% MAC ou Linux, mais cette recherche peut être longue dans certains cas...Du coup,
% on utilise cette solution plus rapide qui éviter à LaTeX de chercher trop longtemps.
%
% /!\ Nécessité d'avoir un fichier fonts/ dans le projet.
%
% /!\ Laisser les options non obligatoires [] avant le nom de la font, autrement
% 	  fontspec n'en fait qu'à sa tête et demande au système où est la police!
\setsansfont[Path=./fonts/, Extension=.otf,
Numbers=OldStyle,
BoldFont=FiraSans-Medium,
ItalicFont=FiraSans-LightItalic,
BoldItalicFont=FiraSans-MediumItalic
]{FiraSans-Light}

\setmainfont[Path=./fonts/,	Extension=.otf,
Numbers=OldStyle,
BoldFont=FiraSans-Medium,
ItalicFont=FiraSans-LightItalic,
BoldItalicFont=FiraSans-MediumItalic
]{FiraSans-Light}