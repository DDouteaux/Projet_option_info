\documentclass{beamer}

% Thème personnalisé
\usetheme{lines}

% Quelques importations pratiques
\usepackage{ifthen}      % pour des confitions dans les commandes
\usepackage{multicol}    % environnement multi colonnes
\usepackage{booktabs}    % pour des tables propres et sans colonnes
\usepackage{pifont}      % des symboles en plus
\usepackage{fontawesome} % utilisation de police d'icônes web

% Les variables générales proposées par Beamer
\title{Projet d'option informatique}
\subtitle{Deep Learning, traitement de langues naturelles et text mining}
\author{Damien \textsc{Douteaux} -- Vincent \textsc{Hocquemiller} -- Louis \textsc{Redonnet}}
\date{Jeudi 30 mars 2017}

% Des commandes ponctuelles pour la présentation
\newcommand{\plus}{\textcolor{vertforet}{\faPlus{}}}
\newcommand{\moins}{\textcolor{bordeau}{\faMinus{}}}
\newcommand{\neutral}{\textcolor{bluenight}{\ding{70}}}

% Contenu de la présentation
% Le choix est ici de séparer chaque section dans un fichier contenu
% dans le répertoire text/. Cela permet d'y voir plus clair et de s'y
% retrouver plus vite dans le document
\begin{document}

% Slide de titre
\begin{frame}[plain]
	\titlepage
\end{frame}

% Sommaire
\section{Sommaire}
\begin{frame}
	\frametitle{Sommaire}
	\tableofcontents
\end{frame}

% Conclusion du rapport
\section{Conclusion}
\begin{frame}
	\frametitle{Conclusion}
	\begin{itemize}
		\setlength{\itemsep}{1.2\baselineskip}
		\item Quasi-respect du calendrier.
		\item Le sujet final est fixé.
		\item Des bases de données repérées et en cours d'étude.
		\item Un début de réflexion sur le réseau de neurones.
	\end{itemize}
\end{frame}

\begin{frame}
	\frametitle{Questions}
	\fontsize{20pt}{20pt}\selectfont
	\vspace*{\fill}
	\begin{center}
		\textbf{Merci pour votre attention}
	\end{center}
	\begin{center}
		\noindent\textbf{Et place aux questions !}
	\end{center}
	\vspace*{\fill}
\end{frame}

\end{document}
